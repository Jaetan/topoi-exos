\documentclass[10pt,a4paper,twoside]{article}
\usepackage{mathtools}
\usepackage{amsfonts}
\usepackage{amssymb}
\usepackage{amsthm}
\usepackage{mathrsfs}
\usepackage{cleveref}
\usepackage[shortlabels]{enumitem}
\usepackage{parskip}
\usepackage{tikz-cd}

\DeclareMathOperator{\dom}{dom}
\DeclareMathOperator{\cod}{cod}

\newcommand{\aname}[1]{{\normalfont\textbf{#1}}}
\newcommand{\Id}{\aname{1}}

%% \newcommand{\catname}[1]{{\normalfont\textbf{#1}}}
%% \newcommand{\Set}{\catname{Set}}

\theoremstyle{definition}
\newcounter{excounter}
\setcounter{excounter}{0}
\newtheorem{exercise}[excounter]{Exercise}

\begin{document}

\begin{exercise}

  Every identity arrow is iso.

\end{exercise}

\begin{proof}

  Let $\Id_a$ be the identity arrow for an object $a$. The identity laws applied to arrows $f, g \colon a \to a$ give:
  \begin{align*}
    \Id_a \circ f = \Id_a &\quad g \circ \Id_a = \Id_a
  \end{align*}
  Setting $g = \Id_a$ in the first equality and $f = \Id_a$ in the second yields
  \begin{align*}
    g \circ f = \Id_a &\quad g \circ f = \Id_a
  \end{align*}
  which is the definition of $f$ and $g$ being iso from $a$ to $a$. Since $f = g = \Id_a$, we conclude that $\Id_a$ is iso.

\end{proof}

\end{document}
