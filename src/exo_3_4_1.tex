\documentclass[11pt,a4paper,twoside]{article}
\usepackage{mathtools}
\usepackage{amsfonts}
\usepackage{amssymb}
\usepackage{amsthm}
\usepackage{mathrsfs}
\usepackage{cleveref}
\usepackage[shortlabels]{enumitem}
\usepackage{parskip}

\DeclareMathOperator{\dom}{dom}
\DeclareMathOperator{\cod}{cod}

\newcommand{\aname}[1]{{\normalfont\textbf{#1}}}
\newcommand{\Id}{\aname{1}}

%% \newcommand{\catname}[1]{{\normalfont\textbf{#1}}}
%% \newcommand{\Set}{\catname{Set}}

\theoremstyle{definition}
\newcounter{excounter}
\setcounter{excounter}{0}
\newtheorem{exercise}[excounter]{Exercise}

\begin{document}

\begin{exercise}

  For any $\mathscr{C}$-objects, show that:
  \begin{enumerate}[i)]
  \item $a \cong a$
  \item if $a \cong b$, then $b \cong a$
  \item if $a \cong b$ and $b \cong c$, then $a \cong c$
  \end{enumerate}

\end{exercise}

\begin{proof}\hfill

  \begin{enumerate}[i)]
  \item The law of identity for $a$ gives $\Id_a \circ \Id_a = \Id_a$, showing that $\Id_a$ is an isomorphism $a \to a$, which implies that $a \cong a$.
  \item If $a \cong b$, then let $a \xrightarrow{f} b$ be an isomorphism from $a$ to $b$. Then there exists and arrow $b \to a$ noted $f^{-1}$ such that $f \circ f^{-1} = \Id_b$ and $f^{-1} \circ f = \Id_a$. This in turn means that $f^{-1}$ is an isomorphism, so that $b \cong a$.
  \item Let $f$ (resp. $g$) be an isomorphism from $a$ to $b$ (resp. from $b$ to $a$). Then $g \circ f$ is an isomorphism from $a$ to $c$ (its inverse is $f^{-1} \circ g^{-1}$), so that $a \cong c$.
  \end{enumerate}

\end{proof}

\end{document}
