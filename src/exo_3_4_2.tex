\documentclass[11pt,a4paper,twoside]{article}
\usepackage{mathtools}
\usepackage{amsfonts}
\usepackage{amssymb}
\usepackage{amsthm}
\usepackage{mathrsfs}
\usepackage{cleveref}
\usepackage[shortlabels]{enumitem}
\usepackage{parskip}

\DeclareMathOperator{\dom}{dom}
\DeclareMathOperator{\cod}{cod}

\newcommand{\aname}[1]{{\normalfont\textbf{#1}}}
\newcommand{\Id}{\aname{1}}

\newcommand{\catname}[1]{{\normalfont\textbf{#1}}}
\newcommand{\Set}{\catname{Set}}
\newcommand{\Finord}{\catname{Finord}}

\theoremstyle{definition}
\newcounter{excounter}
\setcounter{excounter}{1}
\newtheorem{exercise}[excounter]{Exercise}

\begin{document}

\begin{exercise}

  $\Finord$ is a skeletal category.

\end{exercise}

\begin{proof}

  Let $n$ and $m$ be two objects of $\Finord$, and $f$ an isomorphism from $n$ to $m$. Then $f$ is a bijection between the ordinals $n$ and $m$. If $n \subset m$, then $n$ is a section of $m$ and such a bijection does not exist. From this we deduce that we must have $n = m$ (as the roles of $n$ and $m$ can be exchanged in the previous sentence). In turn, this means that $f = \Id_n = \Id_m$, so that $\Finord$ is skeletal.

\end{proof}

\end{document}
