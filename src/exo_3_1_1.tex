\documentclass[10pt,a4paper,twoside]{article}
\usepackage{mathtools}
\usepackage{amsfonts}
\usepackage{amssymb}
\usepackage{amsthm}
\usepackage{mathrsfs}
\usepackage{cleveref}
\usepackage[shortlabels]{enumitem}
\usepackage{parskip}
\usepackage{tikz-cd}

\DeclareMathOperator{\dom}{dom}
\DeclareMathOperator{\cod}{cod}

%% \newcommand{\catname}[1]{{\normalfont\textbf{#1}}}
%% \newcommand{\Set}{\catname{Set}}

\theoremstyle{definition}
\newcounter{excounter}
\setcounter{excounter}{0}
\newtheorem{exercise}[excounter]{Exercise}

\begin{document}

\begin{exercise}

  In any category,
  \begin{enumerate}
  \item $g \circ f$ is monic if both $f$ and $g$ are monic.
  \item If $g \circ f$ is monic then so is $f$.
  \end{enumerate}

\end{exercise}

\begin{proof}\hfill

  \begin{enumerate}

  \item Suppose that both $f$ and $g$ are monic, and let $h$, $k$, $a$, $b$, $c$, $d$ be arrows and objetcs as in the diagram below:
    \begin{equation}\label{diag:monic}
      \begin{tikzcd}
        a \arrow[r, shift left, "h"] \arrow[r, shift right, "k" below]
        & b \arrow[r, "f"]
        & c \arrow[r, "g"]
        & d
      \end{tikzcd}
    \end{equation}

    Suppose that $g \circ f \circ h = g \circ f \circ k$. Then we have:
    \begin{align*}
      g \circ ( f \circ h ) &= g \circ ( f \circ k ) &\text{associativity of }\circ \\
                  f \circ h &= f \circ k             &\text{since $g$ is monic} \\
                          h &= k                     &\text{since $f$ is monic}
    \end{align*}
    From this we deduce that $g \circ f$ is monic.

  \item Suppose that there are objects and arrows as shown in diagram \ref{diag:monic}, that $g \circ f$ is monic, and that $f \circ h = f \circ k$. Then we have:
    \begin{align*}
      f \circ h             &= f \circ k \\
      g \circ ( f \circ h ) &= g \circ ( f \circ k ) &\text{since }\cod(f) = \dom(g) \\
      ( g \circ f ) \circ h &= ( g \circ f ) \circ k &\text{associativity of }\circ \\
                          h &= k                     &\text{since $g \circ f$ is monic}
    \end{align*}
    From this we deduce that $f$ is itself monic.

  \end{enumerate}

\end{proof}

\end{document}
