\documentclass[11pt,a4paper,twoside]{article}
\usepackage{mathtools}
\usepackage{amsfonts}
\usepackage{amssymb}
\usepackage{amsthm}
\usepackage{mathrsfs}
\usepackage{cleveref}
\usepackage[shortlabels]{enumitem}
\usepackage{parskip}

\DeclareMathOperator{\dom}{dom}
\DeclareMathOperator{\cod}{cod}

\newcommand{\aname}[1]{{\normalfont\textbf{#1}}}
\newcommand{\Id}{\aname{1}}

%% \newcommand{\catname}[1]{{\normalfont\textbf{#1}}}
%% \newcommand{\Set}{\catname{Set}}

\theoremstyle{definition}
\newcounter{excounter}
\setcounter{excounter}{1}
\newtheorem{exercise}[excounter]{Exercise}

\begin{document}

\begin{exercise}

If $f$ is iso, so is $f^{-1}$.

\end{exercise}

\begin{proof}

  An arrow $a \xrightarrow{F} b$ is iso iff there exists an arrow $b \xrightarrow{G} a$ such that
  \begin{align}
    F \circ G = \Id_b &\quad G \circ F = \Id_a
  \end{align}

  For $a \xrightarrow{f} b$ an iso arrow, the above equations are satisfied if we set $F = f$ and $G = f^{-1}$. But they are also satisfied if we set $F = f^{-1}$ and $G = f$, so that $f^{-1}$ is an iso arrow $b \to a$ with $f$ as its inverse.

\end{proof}

\end{document}
