\documentclass[11pt,a4paper,twoside]{article}
\usepackage{mathtools}
\usepackage{amsfonts}
\usepackage{amssymb}
\usepackage{amsthm}
\usepackage{mathrsfs}
\usepackage{cleveref}
\usepackage[shortlabels]{enumitem}
\usepackage{parskip}
\usepackage{tikz-cd}

\DeclareMathOperator{\dom}{dom}
\DeclareMathOperator{\cod}{cod}

\newcommand{\aname}[1]{{\normalfont\textbf{#1}}}
\newcommand{\Id}{\aname{1}}

\newcommand{\catname}[1]{{\normalfont\textbf{#1}}}
\newcommand{\Set}{\catname{Set}}

\theoremstyle{definition}
\newcounter{excounter}
\setcounter{excounter}{0}
\newtheorem{exercise}[excounter]{Exercise}

\begin{document}

\begin{exercise}

  Prove that all terminal $\mathscr{C}$-objects are isomorphic.

\end{exercise}

\begin{proof}

  Let $a$ and $b$ be terminal $\mathscr{C}$-objects. Then the identity $\Id_a \colon a \to a$ (resp. $\Id_b \colon b \to b$) is the unique arrow $a \to a$ (resp. $b \to b$).
  Since $a$ (resp. $b$) is terminal, there exists a unique arrow $j \colon b \to a$ (resp. $i \colon a \to b$). Then the composite arrow $j \circ i$ (resp. $i \circ j$) is an arrow from $a$ to $a$ (resp. $b$ to $b$), and therefore equal to $\Id_a$ (resp. $\Id_b$). Therefore we have:
  \begin{align*}
    i \circ j = \Id_b && j \circ i = \Id_a
  \end{align*}
  which shows that $a$ and $b$ are isomorphic.

\end{proof}

\end{document}
