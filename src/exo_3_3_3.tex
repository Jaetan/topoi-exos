\documentclass[11pt,a4paper,twoside]{article}
\usepackage{mathtools}
\usepackage{amsfonts}
\usepackage{amssymb}
\usepackage{amsthm}
\usepackage{mathrsfs}
\usepackage{cleveref}
\usepackage[shortlabels]{enumitem}
\usepackage{parskip}

\DeclareMathOperator{\dom}{dom}
\DeclareMathOperator{\cod}{cod}

\newcommand{\aname}[1]{{\normalfont\textbf{#1}}}
\newcommand{\Id}{\aname{1}}

%% \newcommand{\catname}[1]{{\normalfont\textbf{#1}}}
%% \newcommand{\Set}{\catname{Set}}

\theoremstyle{definition}
\newcounter{excounter}
\setcounter{excounter}{2}
\newtheorem{exercise}[excounter]{Exercise}

\begin{document}

\begin{exercise}

  $f \circ g$ is iso if $f$, $g$ are, with $( f \circ g )^{-1} = g^{-1} \circ f^{-1}$.

\end{exercise}

\begin{proof}

  Let $a \xrightarrow{g} b$ and $b \xrightarrow{f} c$ be two iso arrows, with $g^{-1}$ and $f^{-1}$ their respective inverses. Then we have:
  \begin{align*}
    ( f \circ g ) \circ ( g^{-1} \circ f^{-1} ) &= & ( g^{-1} \circ f^{-1} ) \circ ( f \circ g ) &= \\
    f \circ ( g \circ g^{-1} ) \circ f^{-1}     &= & g^{-1} \circ  ( f^{-1} \circ f ) \circ g    &= \\
    f \circ \Id_b \circ f^{-1}                 &= & g^{-1} \circ \Id_b \circ g                  &= \\
    ( f \circ \Id_b ) \circ f^{-1}             &= & ( g^{-1} \circ \Id_b ) \circ g              &= \\
    f \circ f^{-1}                             &= & g^{-1} \circ g                              &= \\
    \Id_c                                      &  & \Id_a
  \end{align*}
  so that $f \circ g$ is iso with inverse $( f \circ g )^{-1} = g^{-1} \circ f^{-1}$.

\end{proof}

\end{document}
